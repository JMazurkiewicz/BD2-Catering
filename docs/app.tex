\documentclass[10pt]{article}
\usepackage[utf8]{inputenc}
\usepackage[T1]{fontenc}
\usepackage[polish]{babel}
\usepackage{hyperref}
\usepackage{graphicx}
\usepackage{array}

\newcommand{\quotes}[1]{``#1''}
\renewcommand{\arraystretch}{1.2}

% Komenda do opisu przypadków użycia
\newcommand{\usecase}[6]{
    \begin{center}
        \begin{tabular}{|p{7em}|p{21em}|}
            \hline
            \textbf{Nazwa} & #1 \\
            \hline
            \textbf{Aktor(zy)} & #2 \\
            \hline
            \textbf{Zdarzenia \newline inicjujące} & #3 \\
            \hline
            \textbf{Warunki \newline początkowe} & #4 \\
            \hline
            \textbf{Scenariusz główny} & #5 \\
            \hline
            \textbf{Scenariusze alternatywne} & #6 \\
            \hline
        \end{tabular}
    \end{center}
}

\title{Katering - projekt aplikacji}

\author{Jakub Mazurkiewicz, Damian Piotrowski, Konrad Wojewódzki}
\date{Maj 2021}

\begin{document}

\maketitle 

\section{Projekt aplikacji}

\subsection{Implementowane przypadki użycia}

\subsubsection{Przypadki użycia dla recepcjonisty}

\usecase{
    Dodawanie zamówienia
}{
    Recepcjonista
}{
    Wybranie opcji \quotes{Dodaj do zamówienia} w menu lub w kalendarzu.
}{
    Aktor jest zalogowany w systemie.
}{
    1. Aktor otwiera menu dodawania zamówienia, \newline
    2. Aktor wprowadza dane klienta do systemu, \newline
    3. Aktor dostosowuje termin zamówienia do potrzeb klienta, \newline
    4. Aktor wybiera dania zamówione przez klienta, \newline
    5. Aktor zatwierdza wprowadzone zamiany.
}{
    1. Brak wolnego terminu - prośba o wybranie nowego terminu, \newline
    2. Błąd połączenia z DBMS - system prosi o ponowienie akcji.
}

\usecase{
    Edytowanie zamówienia
}{
    Recepcjonista
}{
    Wybranie opcji \quotes{Edytuj zamówienie} w kalendarzu.
}{
    Aktor jest zalogowany w systemie. \newline
    Zamówienie istnieje w systemie.
}{
    1. Aktor otwiera kalendarz, wybiera zamówienie i wybiera opcję \quotes{Edytuj}, \newline
    2. Aktor edytuje zamówienie zgodnie z wolą klienta - np. zamienia termin, dania, \newline
    3. Aktor zatwierdza wprowadzone zamiany.
}{
    1. Wybrany termin jest niedostępny - prośba o wybranie nowego terminu, \newline
    2. Błąd połączenia z DBMS - system prosi o ponowienie akcji.
}

\usecase{
    Usuwanie zamówienia
}{
    Recepcjonista
}{
    Wybranie opcji \quotes{Usuń zamówienie} w kalendarzu.
}{
    Aktor jest zalogowany w systemie. \newline
    Zamówienie istnieje w systemie.
}{
    1. Aktor otwiera kalendarz, wybiera zamówienie i wybiera opcję \quotes{Usuń}, \newline
    2. Aktor zatwierdza usunięcia zamówienie (ewentualne potwierdzenie akcji hasłem), \newline
    3. Aktor zatwierdza wprowadzone zamiany.
}{
    Błąd połączenia z DBMS - system prosi o ponowienie akcji.
}

\usecase{
    Sprawdzanie dostępności terminu
}{
    Recepcjonista
}{
    Otworzenie kalendarza.
}{
    Aktor jest zalogowany w systemie.
}{
    1. Aktor otwiera kalendarz, \newline
    2. Na podstawie widoku kalendarza aktor określa czy dany termin jest wolny.
}{
    1. Aktor może też sprawdzić dostępność terminu korzystając z wyszukiwarki w kalendarzu - aktor wprowadza termin i sprawdza czy jest dostępny. \newline
    2. Błąd połączenia z DBMS - system prosi o ponowienie akcji.
}

\usecase{
    Przypisywanie pracowników do wydarzenia
}{
    Recepcjonista
}{
    Wybranie opcji \quotes{Dodaj pracownika} w kalendarzu.
}{
    Aktor jest zalogowany w systemie. \newline
    Pracownik istnieje w bazie danych.
}{
    1. Aktor otrzymuje listę dostępnych pracowników w terminie danego wydarzenia, \newline
    2. Aktor na podstawie rozmowy telefonicznej z pracownikiem przypisuje go do wydarzenia, \newline
    3. Aktor zatwierdza wprowadzone zamiany.
}{
    1. W przypadku braku dostępnych pracowników w danym terminie aktor wystawia ogłoszenie o poszukiwaniu pracownika. Zakładamy, że szanse na nieznalezienia pracownika są bardzo małe (pomijalne). \newline
    2. Błąd połączenia z DBMS - system prosi o ponowienie akcji.
}

\usecase{
    Dodawanie pracowników do bazy danych
}{
    Recepcjonista
}{
    Wybranie opcji \quotes{Dodaj nowego pracownika} w panelu głównym.
}{
    Aktor jest zalogowany w systemie.
}{
    1. Aktor wypełnia formularz dodawania nowego pracownika, \newline
    2. Aplikacja dokonuje wstępnej walidacji danych (np. poprawność numeru PESEL), \newline
    3. Aktor zatwierdza wprowadzone zamiany, \newline
    4. DBMS dodatkowo sprawdza poprawność wprowadzonych danych.
}{
    1. W przypadku wprowadzenia nieprawidłowych danych (np. nieprawidłowego numeru PESEL) system poprosi o poprawienie danych. \newline
    2. Błąd dotyczący nieprawidłowych danych ze strony DBMS - system prosi o poprawienie danych lub w najgorszym wypadku jest to wewnętrzny błąd systemu. \newline
    3. Błąd połączenia z DBMS - system prosi o ponowienie akcji.
}

\usecase{
    Edytowanie danych pracowników
}{
    Recepcjonista
}{
    Wybranie opcji \quotes{Edytuj dane pracowników} w panelu głównym.
}{
    Aktor jest zalogowany w systemie. \newline
    Pracownik istnieje w bazie danych.
}{
    1. Aktor edytuje dane z formularza pracownika, \newline
    2. Aplikacja dokonuje wstępnej walidacji danych (np. poprawność numeru PESEL), \newline
    3. Aktor zatwierdza wprowadzone zamiany, \newline
    4. DBMS dodatkowo sprawdza poprawność wprowadzonych danych.
}{
    1. W przypadku wprowadzenia nieprawidłowych danych (np. nieprawidłowego numeru PESEL) system poprosi o ich poprawienie. \newline
    2. Błąd dotyczący ze strony DBMS - system prosi o poprawienie danych lub w najgorszym wypadku jest to wewnętrzny błąd systemu. \newline
    3. Błąd połączenia z DBMS - system prosi o ponowienie akcji.
}

\pagebreak % ^^^ Recepcjonista / Szef kuchni i kucharz vvv

\subsubsection{Przypadki użycia dla szefa kuchni i kucharza}

\usecase{
    Uzyskiwanie informacji o zamówieniu
}{
    Szef kuchni, kucharz
}{
    Wybranie opcji \quotes{Informacje o zamówieniu} w kalendarzu.
}{
    Aktor jest zalogowany w systemie. \newline
    Zamówienie istnieje w bazie danych.
}{
    1. Aktor wchodzi w kalendarz w panelu głównym, \newline
    2. Aktor wybiera zamówienie w kalendarzu, po czym wybiera z menu podręcznego opcję \quotes{Informacje o zamówieniu}, \newline
    3. Aktor otrzymuje informacje o zamówieniu - m.in. listę dań.
}{
    1. Błąd połączenia z DBMS - prośba o przeładowanie.
}

\usecase{
    Sprawdzanie aktualnego menu
}{
    Szef kuchni, kucharz
}{
    Wybranie opcji \quotes{Aktualne menu} w panelu głównym.
}{
    Aktor jest zalogowany w systemie.
}{
    1. Aktor otrzymuje widok aktualnego menu z podziałem na kategorie, \newline
    2. Aktor może wybrać danie z menu w celu sprawdzenia potrzebnych składników do jego wykonania.
}{
    1. Błąd połączenia z DBMS - prośba o przeładowanie.
}


\usecase{
    Dodawanie dania do menu
}{
    Szef kuchni
}{
    Wybranie opcji \quotes{Aktualne menu} w panelu głównym.
}{
    Aktor jest zalogowany w systemie.
}{
    1. Aktor otrzymuje widok aktualnego menu z podziałem na kategorie, \newline
    2. Aktor wybiera opcję \quotes{Dodaj element}, \newline
    3. Aktor wypełnia formularz w celu dodania nowego elementu, \newline
    4. Aplikacja sprawdza poprawność wprowadzonych danych, \newline
    5. Aktor zatwierdza wprowadzone zmiany, \newline
    6. Aplikacja wylicza dodatkowe parametry, np. cenę dania, \newline
    7. DBMS dodatkowo sprawdza poprawność wprowadzonych danych.
}{
    1. W przypadku wprowadzenia nieprawidłowych danych aplikacja prosi o ich poprawienie. \newline
    2. Błąd ze strony DBMS - system prosi o poprawienie danych lub w najgorszym wypadku jest to wewnętrzny błąd systemu. \newline
    3. Błąd połączenia z DBMS - system prosi o ponowienie akcji.
}

\usecase{
    Edytowanie dania w menu
}{
    Szef kuchni
}{
    Wybranie opcji \quotes{Aktualne menu} w panelu głównym.
}{
    Aktor jest zalogowany w systemie. \newline
    Danie istnieje w bazie danych.
}{
    1. Aktor otrzymuje widok aktualnego menu z podziałem na kategorie, \newline
    2. Aktor wybiera opcję \quotes{Edytuj element}, \newline
    3. Aktor wypełnia formularz w celu zmienienia danych aktualnego elementu, \newline
    4. Aplikacja sprawdza poprawność wprowadzonych danych, \newline
    5. Aktor zatwierdza wprowadzone zmiany, \newline
    6. Aplikacja wylicza dodatkowe parametry, np. cenę dania na podstawie cen produktów, \newline
    7. DBMS dodatkowo sprawdza poprawność wprowadzonych danych.
}{
    1. W przypadku wprowadzenia nieprawidłowych danych aplikacja prosi o ich poprawienie. \newline
    2. Błąd ze strony DBMS - system prosi o poprawienie danych lub w najgorszym wypadku jest to wewnętrzny błąd systemu. \newline
    3. Błąd połączenia z DBMS - system prosi o ponowienie akcji.
}

\usecase{
    Sprawdzanie stanu magazynu
}{
    Szef kuchni, kucharz
}{
    Wybranie opcji \quotes{Stan magazynu} w panelu głównym.
}{
    Aktor jest zalogowany w systemie.
}{
    1. Aktor po wybraniu opcji \quotes{Stan magazynu} otrzymuje pełną listę produktów znajdujących się w chłodni/magazynie w formie tabeli, \newline
    2. Aktor może sortować i dodawać filtry do otrzymanej tabeli.
}{
    1. Błąd połączenia z DBMS - prośba o przeładowanie.
}

\usecase{
    Edycja stanu magazynu (np. zamawianie produktów)
}{
    Szef kuchni
}{
    Wybranie menu \quotes{Magazyn} w panelu głównym.
}{
    Aktor jest zalogowany w systemie.
}{
    1. Aktor wybiera opcję \quotes{Edytuj} znajdującą się na ekranie, \newline
    2. Aktor zmienia ilość danego produktu, \newline
    3. Aplikacja weryfikuje wprowadzone zmiany, \newline
    4. Zmiany zostają umieszczone w bazie danych.
}{
    1. Błąd połączenia z DBMS - prośba o przeładowanie. \newline
    2. Nieprawidłowe dane - aplikacja prosi o poprawienie błędów.
}

\usecase{
    Edycja listy produktów
}{
    Szef kuchni
}{
    Wybranie menu \quotes{Produkty} w panelu głównym.
}{
    Aktor jest zalogowany w systemie.
}{
    1. Aktor wybiera opcję \quotes{Edytuj} znajdującą się na ekranie, \newline
    2. Aktor dodaje nowy produkt lub edytuje już istniejący, \newline
    3. Aplikacja weryfikuje wprowadzone zmiany, \newline
    4. Zmiany zostają umieszczone w bazie danych.
}{
    1. Błąd połączenia z DBMS - prośba o przeładowanie. \newline
    2. Nieprawidłowe dane - aplikacja prosi o poprawienie błędów.
}

\pagebreak % ^^^ Szef kuchni i kucharz / Dostawca vvv

\subsubsection{Przypadki użycia dla dostawcy}

\usecase{
    Sprawdzanie dań/produktów, które należy załadować do pojazdu
}{
    Dostawca
}{
    Wybranie opcji \quotes{Pokaż zawartość zamówienia}
}{
    Aktor jest zalogowany w systemie. \newline
    Sprawdzane zamówienie istnieje w systemie.
}{
    1. Aktor wybiera wydarzenie w kalendarzu, \newline
    2. Aktor wybiera opcję \quotes{Pokaż zawartość zamówienia} dla danego wydarzenia, \newline
    3. Aktor otrzymuje listę dań, które należy załadować do pojazdu.
}{
    1. Błąd połączenia z DBMS - prośba o przeładowanie.
}


\usecase{
    Sprawdzanie lokalizacji wydarzenia, do którego należy dostarczyć jedzenie
}{
    Dostawca
}{
    Wybranie opcji \quotes{Pokaż lokalizację wydarzenia} w kalendarzu.
}{
    Aktor jest zalogowany w systemie. \newline
    Wydarzenie istnieje w bazie danych.
}{
    1. Aktor wybiera wydarzenie w kalendarzu, \newline
    2. Aktor wybiera opcję \quotes{Pokaż wydarzenia zdarzenia} dla danego wydarzenia, \newline
    3. Aktor otrzymuje odnośnik do przeglądarki - otrzymuje widok z \texttt{Google Maps} razem z nawigacją.
}{
    1. Błąd połączenia z DBMS - prośba o przeładowanie. 
}

\subsubsection{Przypadki użycia dla kelnera}

\usecase{
    Uwzględnianie dodatkowych kosztów zamówienia (np. z powodu uszkodzenia mienia)
}{
    Kelner
}{
    Wybranie opcji \quotes{Uwzględnij dodatkowe koszty} w kalendarzu
}{
    Aktor jest zalogowany w systemie. \newline
    Wydarzenie istnieje w bazie danych.
}{
    1. Aktor wybiera wydarzenie w kalendarzu, \newline
    2. Aktor wybiera opcję \quotes{Uwzględnij dodatkowe koszty}, \newline
    3. Aktor wypełnia formularz - opis sytuacji i kwota, \newline
    4. Aktor wprowadza zmiany do bazy danych, \newline
}{
    1. Błąd połączenia z DBMS - prośba o przeładowanie.
}

\subsubsection{Przypadki dla różnych aktorów}

\usecase{
    Sprawdzanie swojego grafiku
}{
    Dostawca, kelner
}{
    Wybranie opcji \quotes{Grafik} w panelu głównym
}{
    Aktor jest zalogowany w systemie.
}{
    1. Aktor po wybraniu opcji w panelu głównym otrzymuje widok kalendarza, \newline
    2. W kalendarzu zaznaczone są dni i ramy czasowe, w których pomaga przy obsłudze wydarzenia.
}{
    1. Błąd połączenia z DBMS - prośba o przeładowanie.
}

\subsection{Diagram stanów}
\begin{center}
    \includegraphics[scale=0.5448]{app/diagram-stanów-alpha.png}
\end{center}

\pagebreak % ^^^ Diagram stanów / Diagram sekwencji vvv

\subsection{Diagram sekwencji}
\texttt{\href{https://drive.google.com/drive/folders/1AHW_von91aril9y62uCa9vJiyDRlB0I8?usp=sharing}{LINK DO DIAGRAMU SEKWENCJI}}

\subsection{Diagram klas}
\texttt{\href{https://miro.com/app/board/o9J_lL81GTA=/}{LINK DO DIAGRAMU KLAS}}

\section{Projekt ekranów}
\texttt{\href{https://miro.com/welcomeonboard/SLn3IxMquzxConVpUMeYinhwTCstNizbIMqoJ5DmRT7SJ334ayztDhFYqNAvMOP7}{LINK DO MIRO}}

\section{Wybór narzędzi}
\begin{table}[h!]
    \begin{center}
        \begin{tabular}{c|c|c}
            Element & Wybrane rozwiązanie & Wersja \\
            \hline
            Język programowania & \texttt{Python} & 3.9.5 \\
            Biblioteka do pracy z DBMS & \texttt{pyodbc} & 4.0.30 \\
            Biblioteka GUI & \texttt{tkinter} & 8.6 \\
            Mapa & \texttt{Google Maps} & - \\
            Widok kalendarza & \texttt{tkcalendar} & 1.6.1 \\
            Tworzenie wykresów dla raportów & \texttt{matplotlib} & 3.4.2
        \end{tabular}
    \end{center}
\end{table}

\end{document}
