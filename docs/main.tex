\documentclass[10pt,a4paper]{article}
\usepackage[utf8]{inputenc}
\usepackage[T1]{fontenc}
\usepackage{isodate}
\usepackage[polish]{babel}
\usepackage{hyperref}
\newcommand{\quotes}[1]{``#1''}

\title{BD2 - Katering}

\author{Jakub Mazurkiewicz, Damian Piotrowski,\\Konrad Wojewódzki, Przemysław Wieczorkowski}
\date{Kwiecień 2021}

\begin{document}

\maketitle

\section{Dokumentacja wstępna}

\subsection{Przedsiębiorstwo}

\subsubsection{Historia}
Przedsiębiorstwo \quotes{Cabbage Catering} zajmuje się dostarczaniem smacznych i pożywnych posiłków na imprezy okolicznościowe. Przedsiębiorstwo zostało założone w 2015 roku przez Przemysława Kapustkę, którego celem było stworzenie dobrze prosperującej firmy kateringowej. Nie spodziewał się on jednak, że firma urośnie do rozmiarów kateringowego królestwa i będzie realizowała bardzo duże ilości zamówień dla zróżnicowanych grup klientów. Z tego też powodu pojawiła się potrzeba stworzenia systemu komputerowego, który będzie wspomagał przedsiębiorstwo w sprawnym realizowaniu zamówień.

\subsubsection{Zasoby firmy}
Siedziba firmy umiejscowiona jest przy ulicy Urzędniczej 2 w Toruniu. W budynku znajduje się biuro, kuchnia, chłodnia oraz magazyn. W firmie są zatrudnieni: 8 kucharzy w tym 2 szefów kuchni, 3 cukierników, 2 dostawców oraz 15 kelnerów obsługujących gości na wydarzeniach. Przedsiębiorstwo dysponuje dwoma pojazdami transportowymi typu Mercedes Sprinter oraz dwoma pojazdami typu Mercedes AMG G63 dla przedstawicieli. Firma nabywa produkty spożywcze w sieci hurtowni Makro. Dzięki naszej infrastrukturze jesteśmy w stanie obsłużyć wydarzenia nawet do 400 osób. W ofercie znajduje się szeroki wybór dań, w tym dania wegańskie, wegetariańskie, bezglutenowe i tym podobne. 

\subsection{Cel projektu}
Celem projektu jest stworzenie relacyjnej bazy danych do wspomagania obsługi klientów oraz logistyki przedsiębiorstwa. Powstanie także aplikacja, która wspomoże harmonogramowanie zamówień oraz monitorowanie stanu magazynu.

\pagebreak

\section{Etap pierwszy}

\subsection{Model behawioralny}

\subsubsection{Aktorzy w systemie}
\begin{itemize}
    \item \textbf{Pracownik recepcjonista} - przyjmuje od klientów zamówienia na dostarczanie usług kateringowych (za pośrednictwem telefonu) i wprowadza za pośrednictwem SZBD zlecenie do systemu. W razie wypadku informuje klienta o braku możliwości realizacji zamówienia.
    \item \textbf{Szef kuchni} - pobiera z bazy danych informacje o zbliżających się wydarzeniach, sprawdza dostępność produktów na stanie (w magazynie/chłodni) i, w razie potrzeby, zamawia produkty niezbędne do przygotowania potraw. Sprawdza przepisy na zamówione potrawy. Może dodać własne przepisy i modyfikować menu.
    \item \textbf{Kucharz} - sprawdza dostępność produktów i przepisy.
    \item \textbf{Pracownik dostawca} - odczytuje z systemu harmonogram wydarzeń i ustala trasę przejazdu, odbiera produkty z hurtowni i weryfikuje zgodność zamówień ze stanem faktycznym.
    \item \textbf{Pracownik kelner} - sprawdza harmonogram wydarzeń i obsługuje wydarzenie.
    \item \textbf{Klient} - sprawdza dostępność wolnych terminów oraz koszt świadczonych usług i ew. składa zamówienie na wybrane menu (albo samodzielnie ustala listę potraw), określa liczbę gości, datę wydarzenia i lokalizację (ew. precyzuje rodzaj wydarzenia). Podaje podstawowe dane kontaktowe (imię, nazwisko, telefon, ew. e-mail). Dodatkowo może zrezygnować z korzystania z usług firmy/odwołać zaplanowane wydarzenie (najpóźniej na 2 tygodnie przed).
\end{itemize}

\subsubsection{Przypadki użycia}
\noindent\textbf{Złożenie zamówienia}
\begin{enumerate}
    \item Klient składa zamówienie telefonicznie lub osobiście na recepcji
    \item Pracownik wprowadza zlecenie do SZBD
    \item System weryfikuje dostępność terminu
    \item Jeśli termin jest wolny wydarzenie zostaje zapisane w systemie
\end{enumerate}
Alternatywa:
\begin{enumerate}
    \item Punkt pierwszy z przypadku podstawowego
    \item Punkt drugi z przypadku podstawowego
    \item Punkt trzeci z przypadku podstawowego
    \item Jeśli termin jest zajęty wydarzenie zostaje odrzucone
\end{enumerate}
\textbf{Sprawdzenie dostępności produktów}
\begin{enumerate}
    \item Szef kuchni pobiera z systemu listę dań do zlecenia
    \item Pracownik sprawdza w systemie czy w magazynie znajdują się produkty potrzebne do wykonania zlecenia
    \item Jeśli wszystkie produkty znajdują się na stanie zlecenie jest przekazywane do kuchni
\end{enumerate}
Alternatywa:
\begin{enumerate}
    \item Punkt pierwszy z przypadku podstawowego
    \item Punkt drugi z przypadku podstawowego
    \item Jeśli brakuje produktów zostaje złożone zamówienie w hurtowni
\end{enumerate}
\textbf{Przeprowadzenie dostawy:}
\begin{enumerate}
    \item Kurier pobiera z systemu adres i datę dostawy
    \item System oblicza ile samochodów potrzeba do realizacji zamówienia
    \item Pracownik sprawdza kompletność zamówienia
    \item Jeśli zamówienie jest kompletne pracownik dostarcza posiłki
\end{enumerate}
Alternatywa:
\begin{enumerate}
    \item Punkt pierwszy z przypadku podstawowego
    \item Punkt drugi z przypadku podstawowego
    \item Punkt trzeci z przypadku podstawowego
    \item Jeśli zamówienie nie jest kompletne pracownik informuje kuchnię o brakach w zamówieniu
\end{enumerate}

\subsection{Model strukturalny}
\subsubsection{Słownik pojęć}
\begin{itemize}
    \item \textbf{Produkt} - pojedynczy składnik przechowywany w magazynie.
    \item \textbf{Danie} - posiłek do wyboru z naszej karty menu. Może składać się z wielu produktów oraz być różnej wielkości zgodnie z wolą klienta. Jest też udostępniona informacja o alergenach.
    \item \textbf{Magazyn} - miejsce, w którym przechowywane są nasze produkty spożywcze.
    \item \textbf{Zamówienie} - proces wyboru konkretnych dań z naszego menu przez klienta, przygotowania ich w odpowiednich ilościach, wraz z zaleceniami zamawiającego, a następnie dostarczenie ich do wyznaczonego miejsca docelowego.
    \item \textbf{Klient} - podmiot składający zamówienie w naszej firmie. Może to być osoba fizyczna lub zarejestrowana firma.
    \item \textbf{Miejsce zamówienia} - lokalizacja, którą klient wybrał do dostarczenia zamówienia.
\end{itemize}

\subsubsection{Model ER (wersja beta)}
\texttt{\href{https://miro.com/app/board/o9J_lL9uIOo=/}{LINK DO MODELU ER ඞ}}

\subsection{Narzędzia}
\begin{table}[h!]
  \begin{center}
    \begin{tabular}{c|c}
      \textbf{Element} & \textbf{Narzędzie} \\
      \hline
      Storyboard/UML & \texttt{Miro} \\
      Dokumentacja & \makecell{\texttt{Latex (Overleaf)}} \\
      System zarządzania bazą danych & \texttt{PostgreSQL} \\
      Język programowania aplikacji & \texttt{Python} \\
      Chmura & \texttt{Microsoft Azure}
    \end{tabular}
  \end{center}
\end{table}

\end{document}
