\documentclass[10pt,a4paper]{article}
\usepackage[utf8]{inputenc}
\usepackage[T1]{fontenc}
\usepackage{isodate}
\usepackage[polish]{babel}
\newcommand{\quotes}[1]{``#1''}
\usepackage{indentfirst}

\title{BD2 - Katering}

\author{Jakub Mazurkiewicz, Damian Piotrowski,\\Konrad Wojewódzki, Przemysław Wieczorkowski}
\date{Kwiecień 2021}

\begin{document}

\maketitle

\section{Dokumentacja wstępna}

\subsection{Przedsiębiorstwo}

\subsubsection{Historia}
Przedsiębiorstwo \quotes{Cabbage Catering} zajmuje się dostarczaniem smacznych i pożywnych posiłków na imprezy okolicznościowe. Przedsiębiorstwo zostało założone w 2015 roku przez Przemysława Kapustkę, którego celem było stworzenie dobrze prosperującej firmy kateringowej. Nie spodziewał się on jednak, że firma urośnie do rozmiarów kateringowego królestwa i będzie realizowała bardzo duże ilości zamówień dla zróżnicowanych grup klientów. Z tego też powodu pojawiła się potrzeba stworzenia systemu komputerowego, który będzie wspomagał przedsiębiorstwo w sprawnym realizowaniu zamówień.

\subsubsection{Zasoby firmy}
Siedziba firmy umiejscowiona jest przy ulicy Urzędniczej 2 w Toruniu. W budynku znajduje się biuro, kuchnia, chłodnia oraz magazyn. W firmie są zatrudnieni: 8 kucharzy w tym 2 szefów kuchni, 3 cukierników, 2 dostawców oraz 15 kelnerów obsługujących gości na wydarzeniach. Przedsiębiorstwo dysponuje dwoma pojazdami transportowymi typu Mercedes Sprinter oraz dwoma pojazdami typu Mercedes AMG G63 dla przedstawicieli. Firma nabywa produkty spożywcze w sieci hurtowni Makro. Dzięki naszej infrastrukturze jesteśmy w stanie obsłużyć wydarzenia nawet do 400 osób. W ofercie znajduje się szeroki wybór dań, w tym dania wegańskie, wegetariańskie, bezglutenowe i tym podobne. 

\subsection{Cel projektu}
Celem projektu jest stworzenie relacyjnej bazy danych do wspomagania obsługi klientów oraz logistyki przedsiębiorstwa. Powstanie także aplikacja, która wspomoże harmonogramowanie zamówień oraz monitorowanie stanu magazynu.

\pagebreak

\subsection{Narzędzia}
\begin{table}[h!]
  \begin{center}
    \begin{tabular}{c|c}
      \textbf{Element} & \textbf{Narzędzie} \\
      \hline
      Storyboard & \texttt{Miro} \\
      Dokumentacja & \makecell{\texttt{Latex (Overleaf)}} \\
      UML & \texttt{draw.io} \\
      System zarządzania bazą danych & \texttt{PostgreSQL} \\
      Język programowania aplikacji & \texttt{Python} \\
      Chmura & \texttt{Microsoft Azure}
    \end{tabular}
  \end{center}
\end{table}

\end{document}
